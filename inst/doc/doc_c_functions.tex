\documentclass[a4paper,oneside,10pt,DIV=12]{scrartcl}

\usepackage{enumerate}
\usepackage[T1]{fontenc}
\usepackage{inputenc}
\usepackage{times}

\usepackage{amsmath}

\usepackage{setspace}
\setstretch{1.1}

\setlength\parindent{24pt}

\usepackage[hyper,lm]{Rd2}

\usepackage{hyperref}
\hypersetup{
    colorlinks=true,
    linkcolor=blue,
    filecolor=blue,
    urlcolor=blue,
    citecolor=blue
}

% ==============================================================================
% Macros for the description of function arguments
% ==============================================================================
\def\VERBOSE{
	\item[\code{verbose}] toggle, \code{[int]}, \code{1}: verbose (i.e., the
	function prints detailed information to the console), \code{0}: quiet.}
\def\SUBSET#1{
	\item[\code{subset#1}] subset, \code{int array[n]}; with elements in the
		set $\{0, 1\}$, where $1$ signifies that the element is in the subset.}
\def\DATA#1#2#3{
	\item[\code{#1}] #2, \code{double array[#3]}.}
\def\WEIGHTS#1{
	\item[\code{#1}] sampling weights, \code{double array[n]}.
}
\def\CENTER{
	\item[\code{center}] center, \code{double array[p]}.
}
\def\SCATTER{
	\item[\code{scatter}] scatter matrix, \code{double array[p, p]}.
}
\def\DIST{
	\item[\code{dist}] distances, \code{double array[n]}.
}
\def\WORKwb{
	\item[\code{work}] work array, typedef struct
		\code{\LinkA{workarray}{workarray}}.
}
\def\WORKreg{
	\item[\code{work}] work array, typedef struct
		\code{\LinkA{workarray}{workarray2}}.
}
\def\WORKARRAY#1#2#3{
	\item[\code{work#1}] #2, \code{double array[#3]}.
}
\def\WBDATA{
	\item[\code{dat}] data, typedef struct \code{\LinkA{wbdata}{wbdata}}.
}
\def\REGDATA{
	\item[\code{dat}] regression data, typedef struct
		\code{\LinkA{regdata}{regdata}}.
}
\def\EST{
	\item[\code{est}] estimates, typedef struct
		\code{\LinkA{estimate}{estimate}}.
}
\def\SUBSETSIZEm{
	\item[\code{m}] size of the subset, \code{[int]}.
}
\def\COLLECT{
	\item[\code{collect}] size of the initial basic subset, \code{[int]}.
}
\def\MAXITER#1{
	\item[\code{maxiter}] maximum number of iterations, \code{[int]#1}
}
\def\IARRAY#1#2{
	\item[\code{iarray}] #1, \code{int array[#2].}
}

% ==============================================================================
% Document
% ==============================================================================
\begin{document}

\title{\Large Documentation of the \texttt{C} functions}
\subtitle{Weighted BACON algorithms}

\author{{\normalsize Tobias Schoch} \\
\begin{minipage}[t][][t]{\textwidth}
	\begin{center}
	\small{University of Applied Sciences Northwestern Switzerland FHNW} \\
	\small{School of Business, Riggenbachstrasse 16, CH-4600 Olten} \\
	\small{\texttt{tobias.schoch{@}fhnw.ch}}
	\end{center}
\end{minipage}}

\date{{\small \today, version 0.2}}
\maketitle

\vspace{1em}
\hrule
\vspace{-0.5em}
\setcounter{tocdepth}{1}
\tableofcontents
\vspace{0.5em}
\hrule
\vspace{1em}

%===============================================================================
\section{Introduction}
In this report, we document the \code{C} functions underlying the \code{wbacon}
\code{R} package. Only the following methods are exported:
\begin{itemize}
	\item \code{\LinkA{wbacon}{wbacon}} (BACON algorithm for multivariate
		outlier detection)
	\item \code{\LinkA{wbacon\_reg}{wbaconreg}} (BACON algorithm for robust
		linear regression)
	\item \code{\LinkA{wquantile}{wquantile}} (weighted quantile)
\end{itemize}

\noindent All other functions are not exported, hence, they are not callable
from \code{R}. The methodological details of the functions are discussed in the
document ``methods.pdf'' (see package folder \code{inst/doc}).

For ease of referencing, we use the following abbreviations.
\begin{description}
	\item[\small{LAPACK}:] Anderson, E., Z. Bai, C. Bischof, L. S. Blackford,
J. Demmel, J. Dongarra, J. D. Croz, A. Greenhaum, S. Hammerling, A. McKenney,
and D. Sorensen (1999). \textit{LAPACK Users’ Guide}, 3rd ed., Philadelphia:
Society for Industrial and Applied Mathematics (SIAM).
	\item[\small{BLAS}:] Blackford, L. S., A. Petitet, R. Pozo, K. Remington,
R. C. Whaley, J. Demmel, J. Dongarra, I. Duff, S. Hammerling, G. Henry,
M. Heroux, L. Kaufman, and A. Lumsdaine (2002). An updated set of basic linear
algebra subprograms (BLAS), \textit{ACM Transactions on Mathematical Software},
28, 135--151.
\end{description}



%===============================================================================
\section{Exported functions}


%---------------------------------------
\HeaderA{wbacon}{Weighted BACON algorithm for multivariate outlier detection}%
	{wbacon}
\begin{Description}
The function implements a weighted variant of Algorithm 3 of Billor et al.
(2000). It calls a weighted variant of Algorithm 2 of Billor et al. (2000) to
initialize the subset (see \code{\LinkA{initialsubset}{initialsubset}}).
\end{Description}
\begin{Usage}
\begin{verbatim}
void wbacon(double *x, double *w, double *center, double *scatter, double *dist,
    int *n, int *p, double *alpha, int *subset, double *cutoff, int *maxiter,
    int *verbose, int *version2, int *collect, int *success)
\end{verbatim}
\end{Usage}
\begin{Arguments}
	\begin{ldescription}
		\DATA{x}{data}{n, p}
		\WEIGHTS{w}
		\CENTER
		\SCATTER
		\DIST
		\item[\code{n, p}] dimensions, \code{[int]}.
		\item[\code{alpha}] tuning constant, \code{[double]}, it defines the
			$1-\alpha$ quantile of the chi-squared distribution.
		\SUBSET{}
		\item[\code{cutoff}] cutoff threshold, \code{[double]}, i.e. $1-\alpha$
			quantile of the chi-squared distribution.
		\MAXITER{.}
		\VERBOSE
		\item[\code{version2}] toggle, \code{[int]}, defines the method to
			construct the initial subset: \code{1}: ``Version 2'' of Billor et
			al. (2000) is used; \code{0}: ``Version 1'' is used.
		\COLLECT
		\item[\code{success}] indicator, \code{[int]}, \code{1}: algorithm
			converged, \code{0}: failure of convergence.
	\end{ldescription}
\end{Arguments}
\begin{Details}
The \code{subset} is implemented as an \code{int array[n]}. Elements in the
subset are coded \code{1}; otherwise \code{0}. The function makes a copy,
\code{w\_cpy}, of the array \code{w} with sampling weights. This copy is used
in the computations (e.g., \code{weightedmean}) and is modified such that
\code{w\_cpy[i] = 0.0} if \code{subset[i] == 0}.

See \code{methods.pdf} for more details.
\end{Details}
\begin{Dependencies}
	\begin{description}
		\item[internal:]
		\code{\LinkA{wquantile\_noalloc}{wquantilenoalloc}},
		\code{\LinkA{euclidean\_norm2}{euclideannorm2}},
		\code{\LinkA{scatter\_w}{scatterw}},
		\code{\LinkA{mean\_scatter\_w}{meanscatterw}},
		\code{\LinkA{mahalanobis}{mahalanobis}},

		\code{\LinkA{initialsubset}{initialsubset}},
		\code{\LinkA{cutoffval}{cutoffval}}, and
		\code{\LinkA{wbacon\_error}{wbaconerror}}
		%
		\item[external:]
		\code{Rmath.h::\LinkA{qchisq}{qchisq}}
	\end{description}
\end{Dependencies}
\begin{Value}
On return, the following slots are overwritten:
	\begin{ldescription}
		\item[\code{center}] estimated weighted coordinate-wise center
		\item[\code{scatter}] estimated lower triangular matrix of the
			weighted scatter matrix
		\item[\code{dist}] Mahalanobis distance
		\item[\code{subset0}] subset of outlier-free observations
		\item[\code{cutoff}] $1-\alpha$ quantile of the chi-squared
			distribution
		\item[\code{maxiter}] number of iteration required
		\item[\code{success}] convergence or failure of convergence
	\end{ldescription}
\end{Value}

\begin{References}
Billor N., Hadi A.S., Vellemann P.F. (2000). BACON: Blocked Adaptive
Computationally efficient Outlier Nominators. \textit{Computational Statistics
and Data Analysis} 34, pp. 279-298.

B{\'e}guin C., Hulliger B. (2008). The BACON-EEM Algorithm for Multivariate
Outlier Detection in Incomplete Survey Data. \textit{Survey Methodology} 34,
pp. 91-103.
\end{References}

%---------------------------------------
\HeaderA{wbacon\_reg}{Weighted BACON algorithm for robust linear regression}%
	{wbaconreg}
\begin{Description}
The function implements a weighted variant of the Algorithms 4 and 5 of
Billor et al.  (2000).
\end{Description}
\begin{Usage}
\begin{verbatim}
void wbacon_reg(double *x, double *y, double *w, double *resid, double *beta,
    int *subset0, double *dist, int *n, int *p, int *m, int *verbose,
    int *success, int *collect, double *alpha, int *maxiter)
\end{verbatim}
\end{Usage}
\begin{Arguments}
	\begin{ldescription}
		\DATA{x}{design matrix}{n, p}
		\DATA{y}{response}{n}
		\WEIGHTS{w}
		\DATA{resid}{reiduals}{n}
		\SUBSET{0}
		\DATA{dist}{distances/ tis}{n}
		\item[\code{n, p}] dimensions, \code{[int]}.
		\item[\code{m}] size of subset, \code{[int]}.
		\VERBOSE
		\item[\code{success}] \code{1}: successful termination; \code{0}: error,
			did not converge, \code{[int]}.
		\COLLECT
		\item[\code{alpha}] cutoff threshold, \code{[double]}, i.e.
			$1- \alpha$ quantile of the Student $t$-distribution.
		\MAXITER{.}
	\end{ldescription}
\end{Arguments}
\begin{Details}
The regression is computed in two steps. First, we call the weighted BACON
algorithm for multivariate outlier detection (Algorithm 3, see
\code{\LinkA{wbacon}{wbacon}}) on the design matrix \code{x} (Note: the
regression intercept, if there is one, must be dropped). As a result, we obtain
\code{subset} and \code{m}, which are then used as an input to
\code{wbacon\_reg}.

The function \code{wbacon\_reg} calls \code{\LinkA{initial\_reg}{initialreg}}
to initialize the regression. Then, it calls
\code{\LinkA{algorithm\_4}{algorithm4}} and
\code{\LinkA{algorithm\_5}{algorithm5}}.

See \code{methods.pdf} for more details.
\end{Details}
\begin{Dependencies}
	\code{\LinkA{initial\_reg}{initialreg}},
	\code{\LinkA{algorithm\_4}{algorithm4}}, and
	\code{\LinkA{algorithm\_5}{algorithm5}}
\end{Dependencies}
\begin{Value}
On return, the following slots are overwritten:
	\begin{ldescription}
		\item[\code{beta}] regression coefficients
		\item[\code{resid}] residuals
		\item[\code{dist}] distances/ tis
		\item[\code{subset0}] subset of outlier-free observations
		\item[\code{maxiter}] number of iteration required
		\item[\code{success}] convergence or failure of convergence
		\item[\code{x}] is overwritten with the QR factorization as returned
			by \code{LAPACK: dgels}, respectively, \code{LAPACK: dgeqrf}
	\end{ldescription}
\end{Value}
\begin{References}
Billor N., Hadi A.S., Vellemann P.F. (2000). BACON: Blocked Adaptive
Computationally efficient Outlier Nominators. \textit{Computational Statistics
and Data Analysis} 34, pp. 279-298.
\end{References}

%---------------------------------------
\HeaderA{wquantile}{Weighted quantile}{wquantile}
\begin{Description}
Weighted quantile.
\end{Description}
\begin{Usage}
\begin{verbatim}
void wquantile(double *array, double *weights, int *n, double *prob,
    double *result)
\end{verbatim}
\end{Usage}
\begin{Arguments}
	\begin{ldescription}
		\DATA{array}{data}{n}
		\WEIGHTS{weights}
		\item[\code{n}] dimension, \code{int}.
		\item[\code{prob}] probability that defines the quantile,
			\code{double}, such that $0 \leq$\code{prob}$\leq 1$.
		\item[\code{result}] quantile, \code{double}.
	\end{ldescription}
\end{Arguments}
\begin{Details}
	\begin{itemize}
	\item The function is based on a weighted version of the Select (FIND,
		quickselect) algorithm of C.A.R. Hoare with the Bentley and McIlroy
		(1993) 3-way partitioning scheme.
		For very small arrays, we use insertion sort.
	\item For equal weighting, i.e. when all elements in \code{weights} are
		equal, wquantile computes quantiles of type 2 in Hyndman and Fan
		(1996).
	\item (Weighted) Select (and Quicksort) is efficient for large arrays. But
		its overhead can be severe for small arrays; hence, we use insertion
		sort for small arrays; cf. Bentley and McIlroy (1993). The size
		threshold below which insertion sort is used can be specified by
		setting the macro \code{\_n\_quickselect} at compile time; see Sect.
		\ref{ch:wquantile}.
	\end{itemize}

See \code{methods.pdf} for more details.
\end{Details}
\begin{Dependency}
\code{\LinkA{wquantile\_noalloc}{wquantilenoalloc}}
\end{Dependency}
\begin{Value}
On return, \code{result} is overwritten with the weighted quantile.
\end{Value}
\begin{References}\relax
Bentley, J.L. and D.M. McIlroy (1993). Engineering a Sort Function,
\textit{Software - Practice and Experience} 23, pp. 1249-1265.

Hyndman, R.J. and Y. Fan (1996). Sample Quantiles in Statistical Packages,
\textit{The American Statistician 50}, pp. 361-365.
\end{References}



%===============================================================================
\clearpage
\section{Error handling [\texttt{wbacon\_error.c}]}
Error handling refers to the functions that operate on matrices, and which may
fail (e.g., because of rank deficiency). These functions return a value of
typedef enum \code{\LinkA{wbacon\_error\_type}{wbaconerrortype}}. The function
\code{\LinkA{wbacon\_error}{wbaconerror}} can be called to return a human
readable error message.

%---------------------------------------
\HeaderA{wbacon\_error\_type}{Error type \code{[typedef enum]}}{wbaconerrortype}
\begin{ldescription}
	\item[\code{WBACON\_ERROR\_OK}] no error.
	\item[\code{WBACON\_ERROR\_RANK\_DEFICIENT}] matrix is rank deficient.
	\item[\code{WBACON\_ERROR\_NOT\_POSITIVE\_DEFINITE}] matrix is not positive
		definite.
	\item[\code{WBACON\_ERROR\_TRIANG\_MAT\_SINGULAR}] triangular matrix is
		singular.
	\item[\code{WBACON\_ERROR\_CONVERGENCE\_FAILURE}] the algorithm did not
		converge
	\item[\code{[WBACON\_ERROR\_COUNT]}] error count. This is not an actual
		error; it is used for internal purposes.
\end{ldescription}

%---------------------------------------
\HeaderA{wbacon\_error}{Human readable error string}{wbaconerror}
%
\begin{Description}
Returns a human readable error string.
\end{Description}
\begin{Usage}
\begin{verbatim}
	const char* wbacon_error(wbacon_error_type err)
\end{verbatim}
\end{Usage}
\begin{Arguments}
	\begin{ldescription}
		\item[\code{err}] error of typedef enum
			\code{[\LinkA{wbacon\_error\_type}{wbaconerrortype}]}.
	\end{ldescription}
\end{Arguments}
\begin{Value}
Returns a string with a human readable error message.
\end{Value}

%===============================================================================
\clearpage
\section{wBACON [\texttt{wbacon.c}]}
To offer functions with a clean interface, most of the functions use the
typedef struct \code{\LinkA{wbdata}{wbdata}} and
\code{\LinkA{workarray}{workarray}}.

%---------------------------------------
\HeaderA{wbdata}{Data \code{[typedef struct]}}{wbdata}
\begin{ldescription}
	\item[\code{n}] dimension.
	\item[\code{p}] dimension.
	\item[\code{x}] pointer to data, \code{double array[n, p]}.
	\item[\code{w}] pointer to weight, \code{double array[n]}.
	\item[\code{dist}] pointer to distance, \code{double array[n]}.
\end{ldescription}

%---------------------------------------
\HeaderA{workarray}{Work arrays \code{[typedef struct]}}{workarray}
\begin{ldescription}
	\IARRAY{pointer to work array}{n}
	\WORKARRAY{\_n}{pointer to work array}{n}
	\WORKARRAY{\_np}{pointer to work array}{n, p}
	\WORKARRAY{\_pp}{pointer to work array}{pp}
	\WORKARRAY{\_2n}{pointer to work array}{2n}
\end{ldescription}


%---------------------------------------
\clearpage
\subsection*{Internal functions}
\addcontentsline{toc}{subsection}{Internal functions}%

%---------------------------------------
\HeaderB{initialsubset}{Internal function}{initialsubset}
\begin{Description}
Computes the initial subset. This is a weighted variant of Algorithm 2 of Billor
et al. (2000).
\end{Description}
\begin{Usage}
\begin{verbatim}
static wbacon_error_type initialsubset(wbdata *dat, workarray *work,
    double *center, double *scatter, int *subset, int *subsetsize,
    int *verbose, int* collect)
\end{verbatim}
\end{Usage}
\begin{Arguments}
	\begin{ldescription}
		\WBDATA
		\WORKwb
		\CENTER
		\SCATTER
		\SUBSET{}
		\item[\code{subsetsize}] size of \code{subset}, \code{[int]}.
		\VERBOSE
		\COLLECT
	\end{ldescription}
\end{Arguments}
\begin{Dependency}
\code{\LinkA{scatter\_w}{scatterw}}
\end{Dependency}
\begin{Value}
The function returns a \code{\LinkA{wbacon\_error\_type}{wbaconerrortype}}:
the return value is either \code{WBACON\_ERROR\_OK} (i.e., no error) or the
error handed over by
\code{\LinkA{check\_matrix\_fullrank}{checkmatrixfullrank}}.

On return, the following slots are overwritten:
\begin{ldescription}
	\item[\code{dat->w}] elements in the initial subset have $w_i=1$,
		else $w_i=0$
	\item[\code{subset}] subset
	\item[\code{subsetsize}] size of the subset
\end{ldescription}
\end{Value}


%---------------------------------------
\HeaderB{mahalanobis}{Internal function}{mahalanobis}
\begin{Description}
Computes the Mahalanobis distance of the $x_i$'s; see \code{methods.pdf} for
the details.
\end{Description}
\begin{Usage}
\begin{verbatim}
static inline wbacon_error_type mahalanobis(wbdata *dat, double *work_np,
    double *work_pp, double *center, double *scatter)
\end{verbatim}
\end{Usage}
\begin{Arguments}
	\begin{ldescription}
		\WBDATA
		\WORKARRAY{\_np}{work array}{n, p}
		\WORKARRAY{\_pp}{work array}{p, p}
		\CENTER
		\SCATTER
	\end{ldescription}
\end{Arguments}
\begin{Dependencies}
	\begin{description}
		\item[internal:] \code{\LinkA{mean\_scatter\_w}{meanscatterw}}
		\item[external:] \code{LAPACK:dtrsm} and \code{LAPACK:dpotrf}
	\end{description}
\end{Dependencies}
\begin{Value}
The function returns a \code{\LinkA{wbacon\_error\_type}{wbaconerrortype}}:
the return value is either \code{WBACON\_ERROR\_OK} (i.e., no error) or
\code{WBACON\_ERROR\_RANK\_DEFICIENT}.

On return, \code{dat->dist} is overwritten with the Mahalanobis distance.
\end{Value}

%---------------------------------------
\HeaderB{scatter\_w}{Internal function}{scatterw}
\begin{Description}
Computes the weighted scatter matrix.
\end{Description}
\begin{Usage}
\begin{verbatim}
static inline void scatter_w(wbdata *dat, double *work, double *center,
    double *scatter)
\end{verbatim}
\end{Usage}
\begin{Arguments}
	\begin{ldescription}
		\WBDATA
		\CENTER
	\end{ldescription}
\end{Arguments}
\begin{Details}
The weighted \code{scatter} matrix is computed without (re-) computing the
\code{center}.
\end{Details}
\begin{Dependency}
\code{BLAS:dsyrk}
\end{Dependency}
\begin{Value}
On return, \code{scatter} is overwritten with the lower triangular matrix
of the weighted scatter matrix.
\end{Value}

%---------------------------------------
\HeaderB{mean\_scatter\_w}{Internal function}{meanscatterw}
\begin{Description}
Computes the weighted scatter matrix.
\end{Description}
\begin{Usage}
\begin{verbatim}
static inline void mean_scatter_w(wbdata *dat, double *work, double *center,
    double *scatter)
\end{verbatim}
\end{Usage}
\begin{Arguments}
	\begin{ldescription}
		\WBDATA
		\WORKARRAY{}{work array}{n, p}
		\CENTER
		\SCATTER
	\end{ldescription}
\end{Arguments}
\begin{Dependency}
\code{BLAS:dsyrk}
\end{Dependency}
\begin{Value}
On return, \code{scatter} and \code{mean} are overwritten with, respectively,
the lower triangular matrix of the weighted scatter matrix and the weighted
coordinate-wise mean.
\end{Value}

%---------------------------------------
\HeaderB{euclidean\_norm2}{Internal function}{euclideannorm2}
\begin{Description}
Computes the squared Euclidean norm $\Vert \bm x - \bm c \Vert_2^2$, where
$\bm c$ denotes the center.
\end{Description}
\begin{Usage}
\begin{verbatim}
static inline void euclidean_norm2(wbdata *dat, double *work, double *center)
\end{verbatim}
\end{Usage}
\begin{Arguments}
	\begin{ldescription}
		\WBDATA
		\item[\code{work}] work array, \code{double array[n, p]}.
		\CENTER
	\end{ldescription}
\end{Arguments}
\begin{Details}
The implementation follows closely S. Hammarling's \code{dnrm2} function in
LAPACK, which uses a onepass algorithm. The algorithm incorporates some form
of scaling to prevent underflows. Higham (2002, p. 507 and 571) shows that
the return value of the function can only overflow if $\Vert \bm x\Vert_2$
exceeds the largest storable double value. See also Hanson and Hopkins (2017).
\end{Details}
\begin{Value}
On return, \code{dat->dist} is overwritten with the Euclidean norm.
\end{Value}
\begin{References}
Hanson, R.J., and T. Hopkins (2017). Remark on Algorithm 539: A Modern
Fortran Reference Implementation for Carefully Computing the Euclidean Norm,
\emph{ACM Transactions on Mathematical Software} 44, Article 24.

Higham, N.J. (2002). \emph{Accuracy and Stability of Numerical Algorithms},
2nd ed., Philadelphia: Society for Industrial and Applied Mathematics.
\end{References}

%---------------------------------------
\HeaderB{check\_matrix\_fullrank}{Internal function}{checkmatrixfullrank}
\begin{Description}
Check whether the array/ matrix \code{x} has full rank.
\end{Description}
\begin{Usage}
\begin{verbatim}
wbacon_error_type check_matrix_fullrank(double *x, int *p, int decom)
\end{verbatim}
\end{Usage}
\begin{Arguments}
	\begin{ldescription}
		\DATA{x}{data}{p, p}
		\item[\code{p}] dimension, \code{[int]}.
		\item[\code{decom}] toggle, \code{[int]}; \code{1}: the Cholesky
			decomposition of \code{x} is computed, \code{0}: the array/ matrix
			\code{x} is taken as is (and is supposed to be a Cholesky factor).
	\end{ldescription}
\end{Arguments}
\begin{Details}
See \code{methods.pdf} for the details.
\end{Details}
\begin{Dependency}
\code{LAPACK:dpotrf}
\end{Dependency}
\begin{Value}
The function returns a instance of \code{\LinkA{wbacon\_error\_type}{wbaconerrortype}}:
\begin{itemize}
	\item \code{WBACON\_ERROR\_OK} (i.e., no error),
	\item \code{WBACON\_ERROR\_NOT\_POSITIVE\_DEFINITE} or
	\item \code{WBACON\_ERROR\_RANK\_DEFICIENT}.
\end{itemize}
\end{Value}

%---------------------------------------
\HeaderB{cutoffval}{Internal function}{cutoffval}
\begin{Description}
Computes the correction factor used in the determination of the chi-squared
quantile criterion; see \code{methods.pdf} for the details.
\end{Description}
\begin{Usage}
\begin{verbatim}
static inline double cutoffval(int n, int k, int p)
\end{verbatim}
\end{Usage}
\begin{Arguments}
	\begin{ldescription}
		\item[\code{k}] subset size, \code{[int]}.
		\item[\code{n, p}] dimensions, \code{[int]}.
	\end{ldescription}
\end{Arguments}
\begin{Value}
Returns the correction factor.
\end{Value}



%===============================================================================
\clearpage
\section{wBACON\_reg [\texttt{wbacon\_reg.c}]}
To offer functions with a clean interface, most of the functions use the
typedef structs \code{\LinkA{regdata}{regdata}} (see \code{regdata.h}),
\code{\LinkA{estimate}{estimate}}, and \code{\LinkA{workarray}{workarray}}.

%---------------------------------------
\HeaderA{wbdata}{Data \code{[typedef struct]}}{regdata}
\begin{ldescription}
	\item[\code{n}] dimension.
	\item[\code{p}] dimension.
	\item[\code{x}] pointer to the design matrix, \code{double array[n, p]}.
	\item[\code{wx}] pointer to a copy of the design matrix,
		\code{double array[n, p]}.
	\item[\code{y}] pointer to the response, \code{double array[n]}.
	\item[\code{wy}] pointer to a copy of the response, \code{double array[n]}.
	\item[\code{w}] pointer to the sampling weights, \code{double array[n]}.
\end{ldescription}

\noindent \textbf{\sffamily Note.} All slots of the instances of the typedef
struct \code{regdata} are considered immutable, with one exception: \code{wx}
and \code{wy} will be modified.

%---------------------------------------
\HeaderA{estimate}{Estimates \code{[typedef struct]}}{estimate}
\begin{ldescription}
	\item[\code{sigma}] regression scale, \code{double}.
	\item[\code{weight}] pointer to the weights, \code{double array[n]}.
	\item[\code{resid}] pointer to the residuals, \code{double array[n]}.
	\item[\code{beta}] pointer to the regression coefficient,
		\code{double array[p]}.
	\item[\code{dist}] pointer to the distances, \code{double array[n]}.
	\item[\code{L}] pointer to the Cholesky factor, \code{double array[p, p]}.
	\item[\code{xty}] pointer to $X^Ty$, \code{double array[p]}.
\end{ldescription}
\noindent \textbf{\sffamily Note.} The slots of the typedef struct
\code{estimate} reflect the data and parameters of the model fit at the
current stage. The instance \code{est} of \code{estimate} is updated
iteratively.

%---------------------------------------
\HeaderA{workarray}{Work arrays \code{[typedef struct]}}{workarray2}
\begin{ldescription}
	\item[\code{lwork}] determines the size of the array \code{dgles\_work},
		\code{[int]};
	\IARRAY{pointer to work array}{n}
	\WORKARRAY{\_n}{pointer to work array}{n}
	\WORKARRAY{\_np}{pointer to work array}{np}
	\WORKARRAY{\_pp}{pointer to work array}{pp}
	\item[\code{degels\_work}] pointer to \code{double array[lwork]}; this
		array is required by \code{LAPACK:dgels}.
\end{ldescription}
\noindent \textbf{\sffamily Note.} The slots of the typedef struct
\code{workarray} are not (and should not be) used to reference data over
different function calls.

%---------------------------------------
\vspace{2em}
\subsection*{Internal functions}
\addcontentsline{toc}{subsection}{Internal functions}%

%---------------------------------------
\HeaderB{initial\_reg}{Internal function}{initialreg}
\begin{Description}
Initializes the least squares estimate.
\end{Description}
\begin{Usage}
\begin{verbatim}
wbacon_error_type initial_reg(regdata *dat, workarray *work, estimate *est,
    int *subset, int *m, int *verbose)
\end{verbatim}
\end{Usage}
\begin{Arguments}
	\begin{ldescription}
		\REGDATA
		\WORKreg
		\EST
		\SUBSET{}
		\SUBSETSIZEm
		\VERBOSE
	\end{ldescription}
\end{Arguments}
\begin{Details}
See \code{methods.pdf} for more details.
\end{Details}
\begin{Dependencies}
	\code{\LinkA{fitwls}{fitwls}},
	\code{\LinkA{psort\_array}{psortarray}}, and
	\code{\LinkA{compute\_ti}{computeti}}
\end{Dependencies}
\begin{Value}
The function returns a \code{\LinkA{wbacon\_error\_type}{wbaconerrortype}}:
the return value is either \code{WBACON\_ERROR\_OK} (i.e., no error) or
\code{WBACON\_ERROR\_RANK\_DEFICIENT}.

On return, the following slots are overwritten:
\begin{ldescription}
	\item[\code{est->sigma}] regression scale
	\item[\code{est->resid}] residuals
	\item[\code{est->beta}] regression coefficients
	\item[\code{est->dist}] distances/ $t_i$'s
	\item[\code{subset}] initial subset
	\item[\code{m}] size of \code{subset1}
\end{ldescription}
\end{Value}


%---------------------------------------
\HeaderB{algorithm\_4}{Internal function}{algorithm4}
\begin{Description}
Computes a weighted variant of Algorithm 4 of Billor et al. (2000).
\end{Description}
\begin{Usage}
\begin{verbatim}
wbacon_error_type algorithm_4(regdata *dat, workarray *work, estimate *est,
    int *subset0, int *subset1, int *m, int *verbose, int *collect)
\end{verbatim}
\end{Usage}
\begin{Arguments}
	\begin{ldescription}
		\REGDATA
		\WORKreg
		\EST
		\SUBSET{0}
		\SUBSET{1}
		\SUBSETSIZEm
		\VERBOSE
		\COLLECT
	\end{ldescription}
\end{Arguments}
\begin{Details}
See \code{methods.pdf} for more details.
\end{Details}
\begin{Dependencies}
	\begin{description}
		\item[internal:]
		\code{\LinkA{update\_chol\_xty}{updatecholxty}},
		\code{\LinkA{cholesky\_reg}{choleskyreg}},
		\code{\LinkA{compute\_ti}{computeti}}, and
		\code{\LinkA{select\_subset}{selectsubset}}
		\item[external:]
		\code{BLAS:dgemv}
	\end{description}
\end{Dependencies}
\begin{Value}
The function returns a \code{\LinkA{wbacon\_error\_type}{wbaconerrortype}}
either \code{WBACON\_ERROR\_OK} (i.e., no error) or the error handed over by
\begin{itemize}
	\item  \code{\LinkA{update\_chol\_xty}{updatecholxty}} or
	\item \code{\LinkA{compute\_ti}{computeti}}.
\end{itemize}

On return, the following slots are overwritten:
\begin{ldescription}
	\item[\code{est->sigma}] regression scale
	\item[\code{est->resid}] residuals
	\item[\code{est->beta}] regression coefficients
	\item[\code{est->dist}] distances/ $t_i$'s
	\item[\code{subset1}] final subset of Algorithm 4
	\item[\code{m}] size of \code{subset1}
\end{ldescription}
\end{Value}

%---------------------------------------
\HeaderB{algorithm\_5}{Internal function}{algorithm5}
\begin{Description}
Computes a weighted variant of Algorithm 5 of Billor et al. (2000).
\end{Description}
\begin{Usage}
\begin{verbatim}
wbacon_error_type algorithm_5(regdata *dat, workarray *work, estimate *est,
    int *subset0, int *subset1, double *alpha, int *m, int *maxiter,
    int *verbose)
\end{verbatim}
\end{Usage}
\begin{Arguments}
	\begin{ldescription}
		\REGDATA
		\WORKreg
		\EST
		\SUBSET{0}
		\SUBSET{1}
		\item[\code{alpha}] defines the $1-\alpha$ quantile of the Student
			$t$-distribution.
		\SUBSETSIZEm
		\MAXITER{.}
		\VERBOSE
	\end{ldescription}
\end{Arguments}
\begin{Details}
See \code{methods.pdf} for more details.
\end{Details}
\begin{Dependencies}
	\begin{description}
		\item[internal:]
		\code{\LinkA{fitwls}{fitwls}} and
		\code{\LinkA{compute\_ti}{computeti}}
		\item[external:]
		\code{Rmath.h::\LinkA{qt}{qt}}
	\end{description}
\end{Dependencies}
\begin{Value}
The function returns a \code{\LinkA{wbacon\_error\_type}{wbaconerrortype}}:
the return value is either the error handed over by
\code{\LinkA{compute\_ti}{computeti}} or
\begin{itemize}
	\item \code{WBACON\_ERROR\_OK} (i.e., no error) or
	\item \code{WBACON\_ERROR\_CONVERGENCE\_FAILURE} if it does not converge
		in \code{maxiter} iterations.
\end{itemize}

On return, the following slots are overwritten:
\begin{ldescription}
	\item[\code{est->sigma}] regression scale
	\item[\code{est->resid}] residuals
	\item[\code{est->beta}] regression coefficients
	\item[\code{est->dist}] distances/ $t_i$'s
	\item[\code{subset1}] final subset of outlier-free data
	\item[\code{m}] size of \code{subset1}
	\item[\code{maxiter}] number of iterations required
\end{ldescription}
\end{Value}

%---------------------------------------
\HeaderB{select\_subset}{Internal function}{selectsubset}
\begin{Description}
Selects the smallest \code{1..m} observations in \code{x} into the
\code{subset}.
\end{Description}
\begin{Usage}
\begin{verbatim}
void select_subset(double *c, int *iarray, int *subset, int *m, int *n)
\end{verbatim}
\end{Usage}
\begin{Arguments}
	\begin{ldescription}
		\DATA{x}{data}{n}
		\IARRAY{work array}{n}
		\SUBSET{}
		\SUBSETSIZEm
	\end{ldescription}
\end{Arguments}
\begin{Details}
The function calls \code{\LinkA{psort\_array}{psortarray}} to (partially) sort
the elements of \code{x} in ascending order. Then, the smallest $m$
observations are selected into \code{subset}.
\end{Details}
\begin{Value}
On return, \code{subset} is overwritten with the generated subset.
\end{Value}

%---------------------------------------
\HeaderB{compute\_ti}{Internal function}{computeti}
\begin{Description}
Compute the $t_i$'s (\code{tis}) of Billor et al. (2000, p. 288).
\end{Description}
\begin{Usage}
\begin{verbatim}
wbacon_error_type compute_ti(regdata *dat, workarray *work, estimate *est,
    int *subset, int *m, double* tis)
\end{verbatim}
\end{Usage}
\begin{Arguments}
	\begin{ldescription}
		\REGDATA
		\WORKreg
		\EST
		\SUBSET{}
		\SUBSETSIZEm
		\item[\code{tis}] \code{double array[n]}.
	\end{ldescription}
\end{Arguments}
\begin{Details}
The function calls \code{\LinkA{hat\_matrix}{hatmatrix}} to compute the
diagonal elements of the ``hat'' matrix and computes the regression scale.
Then, it computes the $t_i$'s.
\end{Details}
\begin{Dependency}
\code{\LinkA{hat\_matrix}{hatmatrix}}
\end{Dependency}
\begin{Value}
The function returns a \code{\LinkA{wbacon\_error\_type}{wbaconerrortype}}:
the return value is either \code{WBACON\_ERROR\_OK} (i.e., no error) or the
error handed over by \code{\LinkA{hat\_matrix}{hatmatrix}}.

On return, \code{tis} is overwritten with the computed $t_i$'s.
\end{Value}

%---------------------------------------
\HeaderB{cholesky\_reg}{Internal function}{choleskyreg}
\begin{Description}
Compute the least squares estimate using the Cholesky factor $L$ and the
matrix $X^Ty.$
\end{Description}
\begin{Usage}
\begin{verbatim}
void cholesky_reg(double *L, double *x, double *xty, double *beta, int *n,
    int *p)
\end{verbatim}
\end{Usage}
\begin{Arguments}
	\begin{ldescription}
		\item[\code{L}] Cholesky factor, \code{double array[p, p]}.
		\DATA{x}{data}{n}
		\item[\code{xty}] $X^Ty$ \code{double array[p]}.
		\item[\code{beta}] regression coefficients \code{double array[p]}.
		\item[\code{n}] dimension.
		\item[\code{p}] dimension.
	\end{ldescription}
\end{Arguments}
\begin{Value}
On return, \code{beta} is overwritten with the updated least squares estimate.
\end{Value}

%---------------------------------------
\HeaderB{hat\_matrix}{Internal function}{hatmatrix}
\begin{Description}
Computes the diagonal elements of the extended ``hat'' matrix.
\end{Description}
\begin{Usage}
\begin{verbatim}
wbacon_error_type hat_matrix(regdata *dat, workarray *work, double *L,
    double *hat)
\end{verbatim}
\end{Usage}
\begin{Arguments}
	\begin{ldescription}
	\REGDATA
	\WORKreg
	\item[\code{L}] Cholesky factor, \code{double array[p, p]}.
	\item[\code{hat}] hat matrx, \code{double array[n]}.
	\end{ldescription}
\end{Arguments}
\begin{Details}
The diagonal elements of the ``hat'' matrix are computed for the observations
in the subset. For the elements not in the subset, an ``extended hat'' matrix
is computed.
\end{Details}
\begin{Value}
The function returns a \code{\LinkA{wbacon\_error\_type}{wbaconerrortype}}:
the return value is either \code{WBACON\_ERROR\_OK} (i.e., no error) or
\code{WBACON\_ERROR\_TRIANG\_MAT\_SINGULAR} when the triangular matrix is
singular.

On return, \code{hat} is overwritten with the diagonal elements of the
``hat'' matrix.
\end{Value}


%---------------------------------------
\HeaderB{update\_chol\_xty}{Internal function}{updatecholxty}
\begin{Description}
The function up- and downdates the Cholesky factor $L$ and the matrix product
by comparing the two sets \code{subset0} and \code{subset1}.
\end{Description}
\begin{Usage}
\begin{verbatim}
wbacon_error_type update_chol_xty(regdata *dat, workarray *work, estimate *est,
    int *subset0, int *subset1, int *verbose)
\end{verbatim}
\end{Usage}
\begin{Arguments}
	\begin{ldescription}
		\REGDATA
		\WORKreg
		\EST
		\SUBSET{0}
		\SUBSET{1}
		\item[\code{m}] size of the \code{subset1}, \code{[int]}.
		\VERBOSE
	\end{ldescription}
\end{Arguments}
\begin{Details}
The function \code{update\_chol\_xty} compares the sets \code{subset0} and
\code{subset1}. For all elements that are in \code{subset0} but not in
\code{subset1}, it calls \code{\LinkA{chol\_downdate}{choldowndate}}. For all
elements that are not in \code{subset0} but in \code{subset1}, it calls
\code{\LinkA{chol\_update}{cholupdate}}.
\end{Details}
\begin{Value}
The function returns a \code{\LinkA{wbacon\_error\_type}{wbaconerrortype}}:
the return value is either \code{WBACON\_ERROR\_OK} (i.e., no error) or the
error handed over by \code{\LinkA{chol\_downdate}{choldowndate}}.

On return, \code{L} and \code{xty} are overwritten with their updated values.
\end{Value}

%---------------------------------------
\HeaderB{chol\_update}{Internal function}{cholupdate}
\begin{Description}
Rank-one update of the Cholesky factor.
\end{Description}
\begin{Usage}
\begin{verbatim}
void chol_update(double *L, double *u, int *p)
\end{verbatim}
\end{Usage}
\begin{Arguments}
	\begin{ldescription}
		\item[\code{L}] Cholesky factor, \code{double array[p, p]}.
		\item[\code{u}] rank-one update for \code{L},
			\code{double array[p]}.
		\item[\code{p}] dimension.
	\end{ldescription}
\end{Arguments}
\begin{Details}
This function computes a one rank-one update of the Cholesky factor.
\end{Details}
\begin{Value}
On return, \code{L} is overwritten by its updated value.
\end{Value}

%---------------------------------------
\HeaderB{chol\_downdate}{Internal function}{choldowndate}
\begin{Description}
Rank-one downdate of the Cholesky factor.
\end{Description}
\begin{Usage}
\begin{verbatim}
wbacon_error_type chol_downdate(double *L, double *u, int *p)
\end{verbatim}
\end{Usage}
\begin{Arguments}
	\begin{ldescription}
		\item[\code{L}] Cholesky factor, \code{double array[p, p]}.
		\item[\code{u}] rank-one downdate for \code{L},
			\code{double array[p]}.
		\item[\code{p}] dimension.
	\end{ldescription}
\end{Arguments}
\begin{Details}
This function computes a one rank-one downdate of the Cholesky factor. The
attempt to downdate may break down if the Cholesky factor becomes/is not
positive definite. In this case, an error is returned.
\end{Details}
\begin{Value}
The function returns a \code{\LinkA{wbacon\_error\_type}{wbaconerrortype}}:
the return value is either \code{WBACON\_ERROR\_OK} (i.e., no error) or
\code{WBACON\_ERROR\_RANK\_DEFICIENT}.

On return, \code{L} is overwritten by its downdated value.
\end{Value}


%===============================================================================
\clearpage
\section{Weighted least squares [\texttt{fitwls.c}]}
%---------------------------------------
\HeaderA{fitwls}{Weighted least squares}{fitwls}
\begin{Description}
	Returns the least squares estimate, the matrices Q and R of the QR
	factorization, and the residuals of a weighted linear regression.
\end{Description}
\begin{Usage}
\begin{verbatim}
void fitwls(double *x, double *work_x, double *y, double *work_y, double *w,
    double *resid, double *beta0, int *n, int *p, double *work, int *lwork,
    int *info)
\end{verbatim}
\end{Usage}

\begin{Arguments}
	\begin{ldescription}
		\item[\code{x}] design matrix, \code{double array[n, p]}.
		\WORKARRAY{\_x}{work array}{n, p}
		\item[\code{y}] response variable, \code{double array[n]}.
		\WORKARRAY{\_y}{work array}{n}
		\WEIGHTS{w}
		\item[\code{resid}] residuals, \code{double array[n]}.
		\item[\code{beta0}] coefficients, \code{double array[p]}.
		\item[\code{n, p}] dimensions, \code{[int]}.
		\item[\code{work}] work array used in \code{LAPACK:dgels}, which is
			of size \code{double array[lwork]}.
		\item[\code{lwork}] dimension of array \code{work}, \code{[int]}; if
			\code{lwork<1}, the function determines and returns the optimal
			size of \code{lwork}.
		\item[\code{info}] status, \code{[int]}; if successful, \code{info=0};
			otherwise \code{info}$\neq$\code{0}.
	\end{ldescription}
\end{Arguments}
\begin{Details}
The regression coefficients are computed with the \code{LAPACK:dgels}
subroutine using a QR factorization of the weighted design matrix.
\end{Details}
\begin{Dependencies}
\code{LAPACK:dgels} and \code{BLAS:dgemv}
\end{Dependencies}
\begin{Value}
On return, the following slots are overwritten:
	\begin{ldescription}
		\item[\code{beta0}] regression coefficients
		\item[\code{resid}] residuals
		\item[\code{work\_x}] the QR factorization as returned by the
			subroutine \code{LAPACK:dgeqrf}
		\item[\code{info}] status; if successful, \code{info=0}; otherwise
		the computation failed
	\end{ldescription}
\end{Value}

%===============================================================================
\clearpage
\section{Weighted quantile [\texttt{wquantile.c}]}
\label{ch:wquantile}
The following functions are documented in this section:
\begin{itemize}
	\item \code{\LinkA{wquantile\_noalloc}{wquantilenoalloc}}
	\item \code{\LinkA{wselect0}{wselect0}}
	\item some internal functions
\end{itemize}

\noindent The source file \code{wquantile.c} defines two macros:
\begin{ldescription}
	\item[\code{\_n\_quickselect}] threshold to switch from insertion sort to
		a weighted variant of the Select (FIND, quickselect) algorithm,
		default: \code{40} (i.e., for samples smaller than 40, insertion sort
		is used).
	\item[\code{\_n\_ninther}] threshold for choosing the pivotal element,
		default: \code{50}; for samples smaller than 50, the pivot is chosen
		by the median-of-three; for larger samples, Tukey's ninther is used.
\end{ldescription}
\noindent (Weighted) quicksort/ Select(FIND, quickselect) method is efficient
for large arrays. But its overhead can be severe for small arrays; hence,
we use insertion sort for small arrays; cf. Bentley and McIlroy (1993). We
have determined the numerical values by a series of benchmark tests with
\href{https://github.com/google/benchmark}{Google benchmark} on an ordinary
laptop computer (Intel i7 8th generation).

%---------------------------------------
\HeaderA{wquantile\_noalloc}{Weighted quantile without memory allocation}%
	{wquantilenoalloc}
\begin{Description}
The same as \code{wquantile} but without memory allocation.
\end{Description}
\begin{Usage}
\begin{verbatim}
void wquantile_noalloc(double *array, double *weights, double *work, int *n,
    double *prob, double *result)
\end{verbatim}
\end{Usage}
\begin{Arguments}
	\begin{ldescription}
		\DATA{array}{data}{n}
		\WEIGHTS{weights}
		\WORKARRAY{work}{work array}{2*n}
		\item[\code{n}] dimension, \code{[int]}.
		\item[\code{prob}] probability that defines the quantile,
			such that $0 \leq$\code{prob}$\leq 1$, \code{[double]}.
		\item[\code{result}] quantile, \code{[double]}.
	\end{ldescription}
\end{Arguments}
\begin{Details}
See \code{\LinkA{wquantile}{wquantile}}.
\end{Details}
\begin{Dependencies}
\code{\LinkA{wselect0}{wselect0}} and \code{\LinkA{wquant0}{wquant0}}
\end{Dependencies}
\begin{Value}
On return, \code{result} is overwritten with the weighted quantile.
\end{Value}

%---------------------------------------
\HeaderA{wselect0}%
	{Selection of the k-th largest element (k-th order statistic)}{wselect0}
\begin{Description}
Returns the k-th largest element (k-th order statistic); sampling weights
allowed.
\end{Description}
\begin{Usage}
\begin{verbatim}
void wselect0(double *array, double *weights, int lo, int hi, int k)
\end{verbatim}
\end{Usage}
\begin{Arguments}
	\begin{ldescription}
		\DATA{array}{data}{lo..hi}
		\WEIGHTS{weights}
		\item[\code{lo}] lower boundary of arrays, \code{[int]}.
		\item[\code{hi}] upper boundary of arrays, \code{[int]}.
		\item[\code{k}] k-th largest element, such that
			\code{lo}$\leq$ \code{k}$\leq$\code{hi}, \code{[int]}.
	\end{ldescription}
\end{Arguments}
\begin{Details}
See \LinkA{wquantile}{wquantile}.
\end{Details}
\begin{Dependency}
\code{\LinkA{partition\_3way}{partition3}}
\end{Dependency}
\begin{Value}
On return, element \code{array[k]} is in its final sorted position;
\code{weights} is sorted along with \code{array}.
\end{Value}

%---------------------------------------
\HeaderA{insertionselect}{Internal function}{insertionselect}
\begin{Description}
Computes the weighted quantile by sorting all elements in \code{array} in
ascending order (using insertion sort). For small arrays, this can be
considerably faster than quicksort.
\end{Description}
\begin{Usage}
\begin{verbatim}
double insertionselect(double *array, double *weights, int lo, int hi,
    double prob)
\end{verbatim}
\end{Usage}
\begin{Arguments}
	\begin{ldescription}
		\DATA{array}{data}{n}
		\WEIGHTS{weights}
		\item[\code{lo}] lower boundary of arrays, \code{[int]}.
		\item[\code{hi}] upper boundary of arrays, \code{[int]}.
		\item[\code{prob}] probability that defines the quantile,
			\code{double}, such that $0 \leq$\code{prob}$\leq 1$.
	\end{ldescription}
\end{Arguments}
\begin{Dependency}
\code{\LinkA{swap2}{swap2}}
\end{Dependency}
\begin{Value}
On return, element \code{array[k]} is in its final sorted position;
\code{weights} is sorted along with \code{array}.
\end{Value}

%---------------------------------------
\clearpage
\subsection*{Internal functions}
\addcontentsline{toc}{subsection}{Internal functions}%

%---------------------------------------
\HeaderB{wquant0}{Internal function}{wquant0}
\begin{Description}
Workhorse function that computes the weighted quantile recursively; see
\LinkA{wquantile}{wquantile}.
\end{Description}
\begin{Usage}
\begin{verbatim}
void wquant0(double *array, double *weights, double sum_w, int lo, int hi,
    double prob, double *result)
\end{verbatim}
\end{Usage}
\begin{Dependencies}
	\code{\LinkA{insertionselect}{insertionselect}} and
	\code{\LinkA{partition\_3way}{partition3way}}
\end{Dependencies}


%---------------------------------------
\HeaderB{partition\_3way}{Internal function}{partition3way}
\begin{Description}
3-way partitioning scheme of Bentley and McIlroy's (1993) with weights.
\end{Description}
\begin{Usage}
\begin{verbatim}
void partition_3way(double *array, double *weights, int lo, int hi, int *i,
    int *j)
\end{verbatim}
\end{Usage}
\begin{Dependency}
\code{\LinkA{swap2}{swap2}}
\end{Dependency}
\begin{References}
Bentley, J.L. and D.M. McIlroy (1993). Engineering a Sort Function,
\textit{Software - Practice and Experience} 23, pp. 1249-1265.
\end{References}

%---------------------------------------
\HeaderB{choose\_pivot}{Internal function}{choosepivot}
\begin{Description}
Choose pivotal element: for arrays of size < \code{\_n\_ninther}, the
median of three is taken as pivotal element, otherwise Tukey's ninther is
used; see e.g. Bentley and McIlroy (1993).
\end{Description}
\begin{Usage}
\begin{verbatim}
static inline int choose_pivot(double *array, int lo, int hi)
\end{verbatim}
\end{Usage}
\begin{Dependency}
\code{\LinkA{med3}{med3}}
\end{Dependency}
\begin{References}
Bentley, J.L. and D.M. McIlroy (1993). Engineering a Sort Function,
\textit{Software - Practice and Experience} 23, pp. 1249-1265.
\end{References}

%---------------------------------------
\HeaderB{swap2}{Internal function}{swap2}
\begin{Description}
Two elements in \code{array} are swapped (and the corresponding elements in
array \code{weights} are also swapped).
\end{Description}
\begin{Usage}
\begin{verbatim}
static inline void swap2(double *array, double *weights, int i, int j)
\end{verbatim}
\end{Usage}

%---------------------------------------
\HeaderB{med3}{Internal function}{med3}
\begin{Description}
Median-of-three (but without swaps); see e.g. Sedgewick (1997, Chap. 7.5).
\end{Description}
\begin{Usage}
\begin{verbatim}
static inline double med3(double *array, int i, int j, int k)
\end{verbatim}
\end{Usage}
\begin{References}
Sedgewick, R. (1997). \textit{Algorithms in C, Parts 1-4, Fundamentals, Data
Structures, Sorting, and Searching}, Addison-Wesley Longman Publishing Co.,
Inc., 3rd ed.
\end{References}


%===============================================================================
\section{Partial sorting [\texttt{partial\_sort.c}]}

%---------------------------------------
\HeaderA{psort\_array}{Partially sort an array with index}{psortarray}
\begin{Description}
Partially sorts array \code{x} in ascending order; the accompanying
\code{int} array (called \code{index}) is sorted along with the array.
\end{Description}
\begin{Usage}
\begin{verbatim}
void psort_array(double *x, int *index, int n, int k)
\end{verbatim}
\end{Usage}
\begin{Arguments}
	\begin{ldescription}
		\DATA{x}{data}{n}
		\item[\code{index}] index, \code{int array[n]}; the array will be
			overwritten.
		\item[\code{n}] dimension, \code{[int]}.
		\item[\code{k}] value that determines the upper array boundary of
			\code{x[0..k]}, where \code{k} $\leq$ \code{n}, \code{[int]}.
	\end{ldescription}
\end{Arguments}
\begin{Details}
This function is a wrapper for the function
\LinkA{partial\_sort\_with\_index}{partialsortwithindex}.

The function takes care of generating the array \code{index}. The elements of
this array will set up to be \code{0..(n - 1)}.
\end{Details}
\begin{Dependency}
\code{\LinkA{partial\_sort\_with\_index}{partialsortwithindex}}
\end{Dependency}
\begin{Value}
On return, the array \code{x[0..k]} is partially sorted in ascending order;
the array \code{index[0..k]} is sorted along with \code{x[0..k]}.
\end{Value}

%---------------------------------------
\subsection*{Internal functions}
\addcontentsline{toc}{subsection}{Internal functions}%
Most of the internal functions which are called from
\code{\LinkA{psort\_array}{psortarray}} are identical with the internal
functions of \code{\LinkA{wselect0}{wselect0}}. Therefore, we do not
document separately.

%---------------------------------------
\HeaderB{partial\_sort\_with\_index}{Internal function}{partialsortwithindex}
\begin{Description}
Partially sorts a array \code{x} in ascending order; the accompanying
\code{int} array (called \code{index}) is sorted along with the array.
\end{Description}
\begin{Usage}
\begin{verbatim}
void partial_sort_with_index(double *x, int *index, int *lo, int *hi, int *k)
\end{verbatim}
\end{Usage}
\begin{Arguments}
	\begin{ldescription}
		\DATA{x}{data}{lo..hi}
		\item[\code{index}] index, \code{int array[lo..hi]}; the array will be
			overwritten.
		\item[\code{lo, hi}] indices, \code{[int]}, usually \code{lo = 0} and
			\code{hi = n - 1}.
		\item[\code{k}] an \code{[int]} in \code{lo..hi}; determines the k-th largest
			element up to  which \code{x} is to be sorted.
	\end{ldescription}
\end{Arguments}
\begin{Details}
	The array \code{index} must be generated by the caller.
\end{Details}
\begin{Value}
On return, the elements \code{lo..k} in the array \code{x[lo..hi]} are
partially sorted in ascending order; the array \code{index[lo..k]} is sorted
along with \code{x[lo..k]}.
\end{Value}

\end{document}
