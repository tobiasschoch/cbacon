\documentclass[a4paper,oneside,11pt,DIV=12]{scrartcl}

\usepackage[utf8]{inputenc}
\usepackage{graphicx}
\usepackage{amsmath}
\usepackage{bm}

\setkomafont{captionlabel}{\sffamily\bfseries\small}
\setkomafont{caption}{\sffamily}

\usepackage[T1]{fontenc}
\usepackage{times}
\renewcommand{\familydefault}{\rmdefault} 

\usepackage{SweaveCOLOR}
\setkeys{Gin}{width = 0.8\textwidth}

\usepackage{setspace}
\usepackage[longnamesfirst]{natbib}
\usepackage{enumerate}
\usepackage{hyperref}
\hypersetup{
    colorlinks=true,
    linkcolor=blue,
    filecolor=blue,      
    urlcolor=blue,
    citecolor=blue
}

\setlength\parindent{24pt}
\newcommand{\code}[1]{{\texttt{#1}}}


% ==============================================================================
\begin{document}

\title{\Large Vignette: Weighted BACON algorithms}

\author{{\normalsize Tobias Schoch} \\ 
\begin{minipage}[t][][t]{\textwidth}
	\begin{center}
	\small{University of Applied Sciences Northwestern Switzerland FHNW} \\
	\small{School of Business, Riggenbachstrasse 16, CH-4600 Olten} \\
	\small{\texttt{tobias.schoch{@}fhnw.ch}}
	\end{center}
\end{minipage}} 

\date{{\small \today}}
\maketitle

\shortcites{machler_rousseeuw_etal_2020}

%------------------------------------------------------------------------------
\section{Introduction}\label{sec:introduction}
\setstretch{1.1}

The package \code{wbacon} implements a weighted variant of the BACON algorithms
\citep{billor_hadi_etal_2000} for multivariate outlier detection and robust
linear regression. The extension of the BACON algorithm for outlier detection
to allow for weighting is due to \citet{beguin_hulliger_2008}. These authors
also have extended the algorithms in other directions (not implemented in
\code{wbacon}). 

Additional information on the BACON algorithms and the inplementation can be
found in the documents:
\begin{itemize}
	\item \code{methods.pdf}: A mathematical description of the algorithms and
		their implementation;
	\item \code{doc\_c\_functions.pdf}: A documentation of the \code{C} 
		functions.
\end{itemize}
\noindent Both documents are filed in the package folder \code{inst/doc/}.

\subsubsection*{Organization of this document}
Section \ref{sec:installation} gives instructions how to install and load the
package.  In Section \ref{sec:multivariate}, we illustrate the application of
the \code{wbacon} algorithm for multivariate outlier detection in two case
studies (\code{bushfire} and \code{philips} data). In Section
\ref{sec:regression}, we study the robust regression estimator 
\code{wbacon\_reg}.


%------------------------------------------------------------------------------
\section{Installation}\label{sec:installation}
Make sure that the package \code{devtools} is installed.\footnote{The
\code{devtools} package can be installed from CRAN by
\code{install.packages("devtools")}.} Then, the \code{wbacon} package can be
pulled and installed from
\href{https://www.github.com/tobiasschoch/cbacon}{www.github.com/tobiasschoch/wbacon}
using 

\begin{Schunk}
\begin{Sinput}
> devtools::install_github("tobiasschoch/wbacon")
\end{Sinput}
\end{Schunk}

\noindent The package contains \code{C} code to be compiled. Users of Microsoft
Windows need an installation of the the R tool chain bundle
\href{https://cran.r-project.org/bin/windows/Rtools/}{\code{rtools40}} to build
the package. 

Once the package has been installed, it can be loaded and attached to the 
current R session by 
\begin{Schunk}
\begin{Sinput}
> library(wbacon)
\end{Sinput}
\end{Schunk}


%===============================================================================
\section{Multivariate outlier detection}\label{sec:multivariate}
In this section, we study multivariate outlier detection for the two datasets
\begin{itemize}
	\item \code{bushfire} data (with sampling weights),
	\item \code{philips} data (without sampling weights).
\end{itemize}

\subsection{Bushfire data}
The \code{bushfire} dataset is on satellite remote sensing. These data were used by Campbell (1984)\footnote{Campbell, N.A. (1989). Bushfire Mapping using NOAA AVHRR Data. Technical Report. Commonwealth Scientific and Industrial Research Organisation, North Ryde.} to locate bushfire scars. The data are radiometer readings from polar-orbiting satellites of the National Oceanic and Atmospheric Administration (NOAA) which have been collected continuously since 1981. The measurements are taken on five frequency bands or channels. In the near infrared band, it is possible to distinguish vegetation types from burned surface. At visible wavelengths, the vegetation spectra are similar to burned surface. The spatial resolution is rather low (1.1 km per pixel). 

\subsubsection*{Data preparation}
The \code{bushfire} data contain radiometer readings for 38 pixels and have been studied in \citet{maronna_yohai_1995}, \citet{beguin_hulliger_2002}, \cite{beguin_hulliger_2008}, and \cite{hulliger_schoch_2009a}. The data can be obtained from the \code{R} package \code{modi} \citep{hulliger_sterchi_2020}.\footnote{The data are also distributed with the \code{R} package \code{robustbase} \citep{machler_rousseeuw_etal_2020}.} 
\begin{Schunk}
\begin{Sinput}
> data(bushfire, package = "modi")
\end{Sinput}
\end{Schunk}
\noindent The first 6 readings on the five frequency bands (variables) are
\setstretch{1.0}
\begin{Schunk}
\begin{Sinput}
> head(bushfire)
\end{Sinput}
\begin{Soutput}
   X1  X2  X3  X4  X5
1 111 145 188 190 260
2 113 147 187 190 259
3 113 150 195 192 259
4 110 147 211 195 262
5 101 136 240 200 266
6  93 125 262 203 271
\end{Soutput}
\end{Schunk}
\setstretch{1.1}
\noindent \citet{beguin_hulliger_2008} generated a set of sampling weights. The weights can be attached to the current session by
\begin{Schunk}
\begin{Sinput}
> data(bushfire.weights, package = "modi")
\end{Sinput}
\end{Schunk}

\subsubsection*{Outlier detection}

\setstretch{1.0}
\begin{Schunk}
\begin{Sinput}
> fit <- wBACON(bushfire, w = bushfire.weights, alpha = 0.95)
> fit
\end{Sinput}
\begin{Soutput}
Weighted BACON: Robust location, covariance, and distances
Initialized by method: V2 
Converged in 3 iterations (alpha = 0.95)
\end{Soutput}
\end{Schunk}
\setstretch{1.1}

%FIXME:
\noindent The argument \code{alpha} determines the $(1-\alpha)$-quantile $\chi_{\alpha,d}^2$ of the chi-square distribution with $d$ degrees of freedom.\footnote{The degrees of freedom $d$ is a function of the number of variables $p$, the number of observations $n$, and the size of the current subset $m$; see \code{methods.pdf} in the \code{inst/doc} folder of the package.} All observations whose Mahalanobis distances are smaller than $\chi_{\alpha, d}^2$ are selected into the subset of outlier-free data. It is recommended to choose \code{alpha} on grounds of an educated guess of the share of ``good'' observations in the data. Here, we guessed that 95\% of the observations are not outliers. In general, the choice of \code{alpha} does not exert great influence on the result. For instance, the specifications \code{alpha = 0.95}, \code{alpha = 0.9}, and \code{alpha = 0.8} yield the same result.

By default, the initial subset is determined by the Euclidean norm (initialization method: \code{version = "V2"}). This initialization method is robust because it is based on the coordinate-wise (weighted) median but the resulting estimators of center and scatter are \emph{not affine equivariant}. Let $T(\cdot)$ denote an estimator of a parameter of interest (e.g., covariance matrix) and let $\bm X$ denote the $(n \times p)$ data matrix. An estimator $T$ is affine equivariant if and only if 
\begin{equation*}
	T(\bm A \bm X + \bm b) = \bm A T(\bm X) + \bm b, 
\end{equation*}
\noindent for any nonsingular $(m \times n)$ matrix $\bm A$ and any $n$-vector $\bm b$. Although version \code{"V2"} of the BACON method leads to estimators that are not affine equivariant in the above sense, \citet{billor_hadi_etal_2000} point out that the method is nearly affine equivariant. There exists an alternative initialization method (\code{"version = V1"}) which is based on the coordinate-wise (weighted) means; therefore, it is affine equivariant but \emph{not robust}. 

From the above output, we see that the algorithm converged in three iterations. In case the algorithm does not converge, we may increase the maximum number of iterations (default: \code{maxiter = 50}) and toggle \code{verbose = TRUE} to (hopefully) learn more why the method did not converge.  

In the next step, we want to study the result in more detail. In particular, we are interested in the estimated center and scatter (or covariance) matrix. To this end, we can call the \code{summary()} method on the object \code{fit}. 
\setstretch{1.0}
\begin{Schunk}
\begin{Sinput}
> summary(fit)
\end{Sinput}
\begin{Soutput}
Weighted BACON: Robust location, covariance, and distances
Initialized by method: V2 
Converged in 3 iterations (alpha = 0.95)

Number of detected outliers: 24 (63.16%)

Robust estimate of location:
   X1    X2    X3    X4    X5 
108.0 148.9 274.8 218.2 279.4 

Robust estimate of covariance:
        X1     X2      X3     X4     X5
X1   391.3  303.5 -1410.5 -284.5 -240.1
X2   303.5  262.4  -935.3 -166.5 -147.6
X3 -1410.5 -935.3  7343.3 1765.2 1413.1
X4  -284.5 -166.5  1765.2  467.7  365.0
X5  -240.1 -147.6  1413.1  365.0  287.7

Distances (cutoff: 4.559):
   Min. 1st Qu.  Median    Mean 3rd Qu.    Max. 
  1.348   1.958   2.744   7.827  12.979  23.128 
\end{Soutput}
\end{Schunk}
\setstretch{1.1}

\noindent The method detected 24 outliers. The method \code{is\_outlier()} returns a vector of logicals whether an observation has been flagged as an outlier. 
\begin{Schunk}
\begin{Sinput}
> which(is_outlier(fit))
\end{Sinput}
\begin{Soutput}
 [1]  7  8  9 10 11 12 31 32 33 34 35 36 37 38
\end{Soutput}
\end{Schunk}

\noindent The center and covariance (scatter) matrix can be extracted with the auxiliary functions, respectively, \code{center()} and \code{cov()}.
\begin{Schunk}
\begin{Sinput}
> center(fit)
\end{Sinput}
\begin{Soutput}
      X1       X2       X3       X4       X5 
108.0156 148.8594 274.8438 218.2500 279.4219 
\end{Soutput}
\end{Schunk}

\noindent The robust Mahalanobis distances, whose summary statistic is printed by the \code{summary()} method, can be extracted with the \code{distance()} method. 
An application of this function is the following code snipped 

\setstretch{1.0}
\begin{Schunk}
\begin{Sinput}
> hist(distance(fit), breaks = 20)
> abline(v = fit$cutoff, lty = 2)
\end{Sinput}
\end{Schunk}
\setstretch{1.1}
\noindent the resulting graph is shown in Figure \ref{fig:bushfire_hist}. The vertical dotted line shows the cutoff threshold that has been used by \code{wbacon()} for outlier detection/ nomination. 

\begin{figure}[htb]
\begin{center}
\includegraphics{wbacon-011}
\caption{Histogram of distances from the center (bushfire data)}\label{fig:bushfire_hist}
\end{center}
\end{figure}




%------------------------------------------------------------------------------
\subsection{Philips data}
Old television sets had a cathode ray tube with an electron gun. The emitted beam runs through a diaphragm that lets pass only a partial beam to the screen. The diaphragm consists of 9 components. The Philips data set contains $n = 667$ measurements on the $p = 9$ components (variables); see \citet{rousseeuw_van-driessen_1999}. 

\subsubsection*{Data preparation}
The \code{philips} data can be loaded from the \code{R} package \code{cellWise} \citep{raymaekers_rousseeuw_2020}. These data do not have sampling weights. 
\setstretch{1.0}
\begin{Schunk}
\begin{Sinput}
> data(philips, package = "cellWise")
> head(philips)
\end{Sinput}
\begin{Soutput}
        X1     X2    X3    X4    X5    X6     X7     X8     X9
[1,] 0.153 -0.259 0.140 0.514 2.242 0.443 -0.021 -0.035 -0.065
[2,] 0.119 -0.309 0.132 0.518 2.269 0.458 -0.018 -0.035 -0.053
[3,] 0.173 -0.296 0.138 0.516 2.266 0.461 -0.023 -0.026 -0.052
[4,] 0.135 -0.306 0.139 0.522 2.288 0.464 -0.015 -0.031 -0.051
[5,] 0.143 -0.278 0.139 0.519 2.284 0.465 -0.016 -0.018 -0.054
[6,] 0.140 -0.284 0.159 0.531 2.287 0.465 -0.004 -0.024 -0.052
\end{Soutput}
\end{Schunk}
\setstretch{1.1}


\subsection*{Outlier detection}
We compute the BACON algorithm but this time with the initialization method \code{version = "V1"}.  
\setstretch{1.0}
\begin{Schunk}
\begin{Sinput}
> fit <- wBACON(philips, alpha = 0.99, version = "V1")
> fit
\end{Sinput}
\begin{Soutput}
Weighted BACON: Robust location, covariance, and distances
Initialized by method: V1 
Converged in 9 iterations (alpha = 0.99)
\end{Soutput}
\end{Schunk}
\setstretch{1.1}
\noindent The center of the data is estimated to be
\setstretch{1.0}
\begin{Schunk}
\begin{Sinput}
> print(center(fit), digits = 2)
\end{Sinput}
\begin{Soutput}
    X1     X2     X3     X4     X5     X6     X7     X8     X9 
-0.041 -0.316 -0.051  0.438  2.122  0.434 -0.104 -0.067 -0.089 
\end{Soutput}
\end{Schunk}
\setstretch{1.1}
\noindent and the BACON algorithm detected
\begin{Schunk}
\begin{Sinput}
> sum(is_outlier(fit))
\end{Sinput}
\begin{Soutput}
[1] 132
\end{Soutput}
\end{Schunk}
\noindent outliers.
 
\subsection*{Comparison with MCD}
\citet{rousseeuw_driessen_1999}

\code{covMcd} implemented in the \code{R} package \code{robustbase} of \citet{machler_rousseeuw_etal_2020}


\begin{Schunk}
\begin{Sinput}
> library(robustbase)
> fit_mcd <- covMcd(philips)
\end{Sinput}
\end{Schunk}


\begin{figure}[htb]
\begin{center}
\includegraphics{wbacon-017}
\caption{Distances from the center (philips data)}\label{fig:philips_dist}
\end{center}
\end{figure}




%===============================================================================
\clearpage
\section{Robust linear regression}\label{sec:regression}

\begin{Schunk}
\begin{Sinput}
> data(iris, package = "datasets")
\end{Sinput}
\end{Schunk}


\setstretch{1.0}
\begin{Schunk}
\begin{Sinput}
> reg <- wBACON_reg(Sepal.Length ~ Sepal.Width + Petal.Length + Petal.Width,
+        data = iris)
\end{Sinput}
\begin{Soutput}
Step 0: initial subset, m = 98
Step 1 (Algorithm 4):
  m = 5 (0 up- and 93 downdates)
  m = 6 (4 up- and 3 downdates)
  m = 7 (3 up- and 2 downdates)
  m = 8 (2 up- and 1 downdates)
  m = 9 (1 up- and 0 downdates)
  m = 10 (1 up- and 0 downdates)
  m = 11 (1 up- and 0 downdates)
  m = 12 (1 up- and 0 downdates)
  m = 13 (1 up- and 0 downdates)
  m = 14 (1 up- and 0 downdates)
  m = 15 (1 up- and 0 downdates)
  m = 16 (1 up- and 0 downdates)
Step 2 (Algorithm 5):
  m = 144
  m = 148
  m = 149
\end{Soutput}
\end{Schunk}
\setstretch{1.1}

\noindent and the summary method

\setstretch{1.0}
\begin{Schunk}
\begin{Sinput}
> summary(reg)
\end{Sinput}
\begin{Soutput}
Call:
wBACON_reg(formula = Sepal.Length ~ Sepal.Width + Petal.Length + 
    Petal.Width, data = iris)

Residuals:
     Min       1Q   Median       3Q      Max 
-0.81889 -0.22557  0.01763  0.20347  0.71073 

Coefficients:
             Estimate Std. Error t value Pr(>|t|)    
(Intercept)   1.88038    0.24535   7.664 2.37e-12 ***
Sepal.Width   0.64524    0.06519   9.897  < 2e-16 ***
Petal.Length  0.71056    0.05546  12.813  < 2e-16 ***
Petal.Width  -0.57184    0.12483  -4.581 9.89e-06 ***
---
Signif. codes:  0 '***' 0.001 '**' 0.01 '*' 0.05 '.' 0.1 ' ' 1

Residual standard error: 0.3075 on 145 degrees of freedom
Multiple R-squared:  0.861,	Adjusted R-squared:  0.8582 
F-statistic: 299.5 on 3 and 145 DF,  p-value: < 2.2e-16
\end{Soutput}
\end{Schunk}
\setstretch{1.1}

\setstretch{1.0}
\begin{Schunk}
\begin{Sinput}
> summary(lm(Sepal.Length ~ Sepal.Width + Petal.Length + Petal.Width,
+         data = iris[!is_outlier(reg), ]))
\end{Sinput}
\begin{Soutput}
Call:
lm(formula = Sepal.Length ~ Sepal.Width + Petal.Length + Petal.Width, 
    data = iris[!is_outlier(reg), ])

Residuals:
     Min       1Q   Median       3Q      Max 
-0.81889 -0.22557  0.01763  0.20347  0.71073 

Coefficients:
             Estimate Std. Error t value Pr(>|t|)    
(Intercept)   1.88038    0.24535   7.664 2.37e-12 ***
Sepal.Width   0.64524    0.06519   9.897  < 2e-16 ***
Petal.Length  0.71056    0.05546  12.813  < 2e-16 ***
Petal.Width  -0.57184    0.12483  -4.581 9.89e-06 ***
---
Signif. codes:  0 '***' 0.001 '**' 0.01 '*' 0.05 '.' 0.1 ' ' 1

Residual standard error: 0.3075 on 145 degrees of freedom
Multiple R-squared:  0.861,	Adjusted R-squared:  0.8582 
F-statistic: 299.5 on 3 and 145 DF,  p-value: < 2.2e-16
\end{Soutput}
\end{Schunk}
\setstretch{1.1}




%===============================================================================
% References
\clearpage
\singlespacing
\begin{thebibliography}{9}
\newcommand{\enquote}[1]{``#1''}
\expandafter\ifx\csname natexlab\endcsname\relax\def\natexlab#1{#1}\fi

\bibitem[\protect\citeauthoryear{B{\'e}guin and Hulliger}{B{\'e}guin and
  Hulliger}{2002}]{beguin_hulliger_2002}
\textsc{B{\'e}guin, C. and B.~Hulliger} (2002): \emph{Robust Multivariate
  Outlier Detection and Imputation with Incomplete Survey Data}, {D}eliverable
  D4/5.2.1/2 {P}art {C}: {EUREDIT} project,
  https://www.cs.york.ac.uk/euredit/euredit-main.html, research project funded
  by the {E}uropean {C}ommission, {IST}-1999-10226.

\bibitem[\protect\citeauthoryear{B\'{e}guin and Hulliger}{B\'{e}guin and
  Hulliger}{2008}]{beguin_hulliger_2008}
\textsc{B\'{e}guin, C. and B.~Hulliger} (2008): \enquote{The BACON-EEM
  Algorithm for Multivariate Outlier Detection in Incomplete Survey Data,}
  \emph{Survey Methodology}, Vol. 34, No. 1, 91--103.

\bibitem[\protect\citeauthoryear{Billor, Hadi, and Vellemann}{Billor
  et~al.}{2000}]{billor_hadi_etal_2000}
\textsc{Billor, N., A.~S. Hadi, and P.~F. Vellemann} (2000): \enquote{{BACON}:
  Blocked Adaptative Computationally-efficient Outlier Nominators,}
  \emph{Computational Statistics and Data Analysis}, 34, 279--298.

\bibitem[\protect\citeauthoryear{Hulliger and Schoch}{Hulliger and
  Schoch}{2009}]{hulliger_schoch_2009a}
\textsc{Hulliger, B. and T.~Schoch} (2009): \enquote{Robust multivariate
  imputation with survey data,} in \emph{Proceedings of the 57th Session of the
  International Statistical Institute}, Durban.

\bibitem[\protect\citeauthoryear{Hulliger and Sterchi}{Hulliger and
  Sterchi}{2020}]{hulliger_sterchi_2020}
\textsc{Hulliger, B. and M.~Sterchi} (2020): \emph{modi: Multivariate Outlier
  Detection and Imputation for Incomplete Survey Data}, {R} package version
  0.1-0.

\bibitem[\protect\citeauthoryear{M{\"a}chler, Rousseeuw, Croux, Todorov,
  Ruckstuhl, Salibian-Barrera, Verbeke, Koller, Conceicao, and {Anna di
  Palma}}{M{\"a}chler et~al.}{2020}]{machler_rousseeuw_etal_2020}
\textsc{M{\"a}chler, M., P.~Rousseeuw, C.~Croux, V.~Todorov, A.~Ruckstuhl,
  M.~Salibian-Barrera, T.~Verbeke, M.~Koller, E.~L.~T. Conceicao, and M.~{Anna
  di Palma}} (2020): \emph{robustbase: Basic Robust Statistics}, {R} package
  version 0.93-6.

\bibitem[\protect\citeauthoryear{Maronna and Yohai}{Maronna and
  Yohai}{1995}]{maronna_yohai_1995}
\textsc{Maronna, R.~A. and V.~J. Yohai} (1995): \enquote{The Behavior of the
  Stahel-Donoho Robust Multivariate Estimator,} \emph{Journal of the American
  Statistical Association}, 90, 330--341.

\bibitem[\protect\citeauthoryear{Raymaekers and Rousseeuw}{Raymaekers and
  Rousseeuw}{2020}]{raymaekers_rousseeuw_2020}
\textsc{Raymaekers, J. and P.~Rousseeuw} (2020): \emph{cellWise: Analyzing Data
  with Cellwise Outliers}, {R} package version 2.2.1.

\bibitem[\protect\citeauthoryear{Rousseeuw and {Van Driessen}}{Rousseeuw and
  {Van Driessen}}{1999}]{rousseeuw_van-driessen_1999}
\textsc{Rousseeuw, P.~J. and K.~{Van Driessen}} (1999): \enquote{A fast
  algorithm for the Minimum Covariance Determinant estimator,}
  \emph{Technometrics}, 41, 212--223.
\end{thebibliography}
 
\end{document}
